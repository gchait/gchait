\documentclass{article}
\usepackage[a4paper, total={5in, 11in}]{geometry}
\usepackage{fontspec}
\usepackage{polyglossia}
\usepackage{amsmath}
\usepackage{parskip}
\usepackage{graphicx}
\usepackage{xcolor}

\definecolor{bg} {RGB} {30, 30, 30}
\definecolor{fg} {RGB} {220, 220, 220}
\pagecolor{bg}
\color{fg}
\setmainlanguage{hebrew}
\setmainfont{Arial}
\begin{document}

\title{שלום עולם}
\author{גיא חאיט}
\date{דצמבר 2023}
\maketitle

\section{הראה ש $p \wedge q$ הוא לא סתירה.}

שלום וברכה!

קוראים לי גיא!

$
\begin{array}{|c | c | c|}
     p & q & p \wedge q \\
     \hline
     T & T & T \\
     T & F & F \\
     F & T & F \\
     F & F & F \\
\end{array}
$
\end{document}
